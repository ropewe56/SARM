\section{Results}

In the following some results are presented. 

%TODO
%\fig{fig:toa1} shows the intensity as a function of the wavelength at TOA computed by the above described model excluding emission.   
%\begin{figure}[ht]
% 	\includegraphics[width=70mm]{intensity_at_toa_no_eps.png}
%	\caption{Intensity at the TOA without emission.}
%	\label{fig:toa1}
%\end{figure}

%TODO
%\fig{fig:toa2} shows the intensity versus wavelength at TOA including emission.   
%\begin{figure}[ht]
%	\includegraphics[width=70mm]{intensity_at_toa_with_eps.png}
%	\caption{Intensity at the TOA including emission.}
%	\label{fig:toa2}
%\end{figure}


%\fig{fig:I_vs_h} shows the difference of the intensity integrated from $13 \mu m$ to $17 \mu m$ as a function of the height above earths surface at $400 ppm$ and $800 ppm$ $CO_2$ respectively.   
%\begin{figure}[ht]
%	\includegraphics[width=70mm]{I_vs_h.png}
%	\caption{Intensity vs. height.}
%	\label{fig:I_vs_h}
%\end{figure}
%
%\fig{fig:abs_coeff} and \fig{fig:em_coeff} show the wavelength dependent absorption and emission coefficients. 
%\begin{figure}[ht]
%	\includegraphics[width=70mm]{absorption_coefficient.png}
%	\caption{Absorption coefficient}
%	\label{fig:abs_coeff}
%\end{figure}
%
%\begin{figure}[ht]
%	\includegraphics[width=70mm]{emission_coefficient.png}
%	\caption{Emission coefficient}
%	\label{fig:em_coeff}
%\end{figure}
%
%\fig{fig:int_toa1} shows the intensity over the wavelength at TOA for $400 ppm$ and $800 ppm$ $CO_2$ respectively. 
%\begin{figure}[ht]
%	\includegraphics[width=70mm]{intensity_at_toa.png}
%	\caption{Intensity at TOA}
%	\label{fig:int_toa1}
%\end{figure}
%\newline
%\newline

%\begin{figure}[!ht]
%	\includegraphics[width=70mm]{intensity_at_toa_with_eps.png}
%	\caption{Intensity at TOA including emission}
%	\label{fig:int_toa2}
%\end{figure}


In the following table intensity difference values between $400 ppm$ and $800 ppm$ $CO_2$. 
\newline
\newline
\begin{tabular}{|c |c |c |}
	\hline
	$I(400 bpm) - I(800 bpm) \;\; [W/m^2]$ & including emission & number of isotopes \\
	\hline
	14.8                                     & yes      &  12         \\
	15.3                                     & yes      &  1          \\
	17.4                                     & no       &  1          \\
	12.1                                     & yes      &  1       \\
	\hline   
\end{tabular}  

Flux-diff = $5.0 W/m^2$

\section{Estimating the Temperature Increase}


The power flux from sun to earth outside the atmosphere is (solcar constant):
\begin{align}
	SC = 1367 \; \dfrac{W}{m^2}
\end{align}

The albedo $a$ is approximately $0.3$ so that the necessary mean energy flux 
to space per $m^2$ of earth surface is:
\begin{align}
	F_0 &= (1-a) \dfrac{SC}{4} \\
	F_0 &=  341.75 \dfrac{W}{m^2}
\end{align}

\begin{align}
	F_0 &= \sigma T^4 \\
	F_0 + \Delta F &= \sigma (T+ \Delta T)^4 \\
	\Delta T &= \left(\dfrac{F_0+\Delta F}{\sigma}\right)^{1/4} - T
\end{align}

\begin{align}
	&(F_0 + \Delta F) = \sigma T^4 \left(1 + 4 \dfrac{\Delta T}{T}\right) \\
	&\dfrac{(F_0 + \Delta F)}{\sigma T^4} =\left(1 + 4 \dfrac{\Delta T}{T}\right) \\
	&\Delta T = \left(\dfrac{(F_0 + \Delta F)}{\sigma T^4} - 1\right) \dfrac{T}{4}
\end{align}

$\sigma = 5.670374419 \; 10^{-8} \dfrac{W}{ m^2  K^4}$
