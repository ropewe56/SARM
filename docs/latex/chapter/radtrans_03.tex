\section{Results}

In the following some results are presented. 

figures
%TODO
%\fig{fig:toa1} shows the intensity as a function of the wavelength at TOA computed by the above described model excluding emission.   
%\begin{figure}[ht]
% 	\includegraphics[width=70mm]{intensity_at_toa_no_eps.png}
%	\caption{Intensity at the TOA without emission.}
%	\label{fig:toa1}
%\end{figure}

%TODO
%\fig{fig:toa2} shows the intensity versus wavelength at TOA including emission.   
%\begin{figure}[ht]
%	\includegraphics[width=70mm]{intensity_at_toa_with_eps.png}
%	\caption{Intensity at the TOA including emission.}
%	\label{fig:toa2}
%\end{figure}


%\fig{fig:I_vs_h} shows the difference of the intensity integrated from $13 \mu m$ to $17 \mu m$ as a function of the height above earths surface at $400 ppm$ and $800 ppm$ $CO_2$ respectively.   
%\begin{figure}[ht]
%	\includegraphics[width=70mm]{I_vs_h.png}
%	\caption{Intensity vs. height.}
%	\label{fig:I_vs_h}
%\end{figure}
%
%\fig{fig:abs_coeff} and \fig{fig:em_coeff} show the wavelength dependent absorption and emission coefficients. 
%\begin{figure}[ht]
%	\includegraphics[width=70mm]{absorption_coefficient.png}
%	\caption{Absorption coefficient}
%	\label{fig:abs_coeff}
%\end{figure}
%
%\begin{figure}[ht]
%	\includegraphics[width=70mm]{emission_coefficient.png}
%	\caption{Emission coefficient}
%	\label{fig:em_coeff}
%\end{figure}
%
%\fig{fig:int_toa1} shows the intensity over the wavelength at TOA for $400 ppm$ and $800 ppm$ $CO_2$ respectively. 
%\begin{figure}[ht]
%	\includegraphics[width=70mm]{intensity_at_toa.png}
%	\caption{Intensity at TOA}
%	\label{fig:int_toa1}
%\end{figure}
%\newline
%\newline

%\begin{figure}[!ht]
%	\includegraphics[width=70mm]{intensity_at_toa_with_eps.png}
%	\caption{Intensity at TOA including emission}
%	\label{fig:int_toa2}
%\end{figure}


In the following table intensity differences between $400 \mathrm{ppm}$ and $800 \mathrm{ppm}$  $\mathrm{CO}_2$ 
in the wavelength interval $13-17$ $\mu m$ are shown. (The total radiation flux to space has to be
$239 \;W/m^2$)
\newline

\begin{tabular}{|c |c |c |c |}
	\hline
	szenario & $I(400 \mathrm{bpm})\;\; [W/m^2]$ & $I(800 \mathrm{bpm})\;\; [W/m^2]$ & $I(400 \mathrm{bpm}) - I(800 \mathrm{bpm}) \;\; [W/m^2]$ \\
	\hline
	1 & 54.7 & 49.7 & 5.04 \\
	2 & 43.6 & 37.9 & 5.75 \\
	3 & 39.8 & 34.0 & 5.80 \\
	\hline
\end{tabular}
\begin{itemize}
\item 1 - with emission, T = T(z)
\item 2 - without emission, T = T(z)
\item 3 - without emission, T = const
\end{itemize}

\subsection{Estimating the Temperature Increase}


The power flux from sun to earth outside the atmosphere is (solcar constant):
\begin{align}
	S = 1367 \; \dfrac{W}{m^2}
\end{align}
The albedo $a$ is approximately $0.3$ so that the necessary mean energy flux 
to space per $m^2$ of earth surface is:
\begin{align}
	F_0 &= (1-a) \dfrac{S}{4} =  239 \dfrac{W}{m^2}
\end{align}
The effective radiation temperature is thus:%\cite{Hansen_1984}
\begin{align}
	T_e  = \left(\dfrac{F_0}{\sigma}\right)^{1/4}
\end{align}
To compensate for the reduced flux due to an increaes in $\mathrm{CO}_2$ concentration 
the effective temperature has to increase. A simple estimate is:
\begin{align}
	F_0 + \Delta F &= \sigma (T_e + \Delta T)^4 \\
	\Delta T &= \left(\dfrac{F_0+\Delta F}{\sigma}\right)^{1/4} - T_e
\end{align}
which yields:
\begin{align}
	\Delta T &= \left(\dfrac{F_0+\Delta F}{\sigma}\right)^{1/4} - T_e
\end{align}
With $\Delta F = 5 W/m^2$ it follows $\Delta T = 1.33 \;K$

%\begin{align}
%	&(F_0 + \Delta F) = \sigma T^4 \left(1 + 4 \dfrac{\Delta T}{T}\right) \\
%	&\dfrac{(F_0 + \Delta F)}{\sigma T^4} =\left(1 + 4 \dfrac{\Delta T}{T}\right) \\
%	&\Delta T = \left(\dfrac{(F_0 + \Delta F)}{\sigma T^4} - 1\right) \dfrac{T}{4}
%\end{align}
%
%$\sigma = 5.670374419 \; 10^{-8} \dfrac{W}{ m^2  K^4}$
