\section{Introduction}

\textbf{SARM} is a model for the radiative transfer of IR radiation from the earth's surface to the  top of the atmosphere (TOA) in which only $\mathrm{CO}_2$ interacts with the radiation. It is by no means an accurate model of radiative transport in the atmosphere with its multitude of climate-active gases, clouds, aerosols etc. The current model is only intended to provide some basic insights into the absorption and emission processes of infrared radiation on its way from the earth's surface to TAO. The wavelength range is limited to $(13 - 17)\; \mu m$. Temperature and pressure of the atmosphere are functions of altitude. However, they are not self-consistently calculated in the frame of the model but are assumed to be given.
\section{Model Description}

The key assumptions of \textbf{SARM} are:
\begin{itemize}
	\item The only species interating with infrared radiation in the $15 \mu m$ range is $\mathrm{CO}_2$
	\item The temperature and pressure profiles of the atmosphere are prescribed, thus there is no self consistency between absorption, emission and temperature
	\item Only upward traveling radiation is considered, the model is 1-dimensional
	\item The main result is the difference in the power of infrared radiation at the top of the atmosphere (TOA) at $70 \;km$ at different $\mathrm{CO}_2$ concentrations.
	\item Scattering at areosols, droplets etc. will be neglected
	\item The atmospheric gas is in local thermodynamic equilibrium so that all energy levels are occupied according to the Boltzmann factor
	\item All $\mathrm{CO}_2$  spectroscopic data are taken from the HITRAN data base.
\end{itemize}

\subsection{Problem description}

Most of the energy heating the earth surface comes from the sun. Autside the atmosphere the radiative flux from the sun is:
\begin{itemize}
	S = 1342 \; \dfrac{W}{m^2}
\end{itemize}
A part of this flux is reflected, the is given by the albedio $a$
which is approximately $0.3$.
The total flux absorbed by the earth atmosphere land and aceans then is:
\begin{itemize}
	E_a = S (1-a) \pi R^2
\begin{itemize}
On average this has to be send back to space. Per $m^2$ of the earth surface $A_e = 4 \pi/3 R^2$ this amounts to $\dfrac{E_a}{4}$. 
Assuming black body radiation an effective radiation temperatiure can be estimated:
\begin{itemize}
	T_e = \left(\dfrac{E_a}{\sigma}\right)^{1/4} \approx = 
\end{itemize}
The earth surface temperature $T_s$ has to be high enough so that the flux from the surface after passing through the atmospehere where part of the fradiation is absorbed
at the top of the atmosphere (TOA) the outward flux = $E_a$.

\subsection{Radiation Transport Equation}

The radiation transport equation reads:
\begin{align}
	\label{eqn1}
	\dfrac{d I(\lambda)}{dz} = - \kappa(\lambda) I(\lambda) + \epsilon(\lambda)
\end{align}
with the spectral intensity:
\begin{align*}
	&I(\lambda)    : \left[\dfrac{W}{m^2 \; sr \; m}\right]
\end{align*}
the absorption coefficient:
\begin{align*}
	&\kappa(\lambda)   : \left[\dfrac{1}{m}\right]
\end{align*}
and the emission coefficient:
\begin{align*}
	&\epsilon(\lambda) : \left[\dfrac{W}{m^3 \; sr \; m}\right]
\end{align*}
The absorption and emission coefficients are given by the sums over all 
line coefficients:
\begin{align*}
	\epsilon(\lambda) &= \sum_j \epsilon_j(\lambda) \\
	\kappa(\lambda) &= \sum_j \kappa_j(\lambda)
\end{align*}

The emission coefficient $\epsilon_j(\lambda)$ due to spontaneous emission from the upper level $u$ to the lower level $l$ is given by:
\begin{align}
	\epsilon_j(\lambda) &= \dfrac{1}{4 \pi} \dfrac{h c}{\lambda} N_{u,j} A_{ul,j} f_j(\lambda)
	 \dfrac{\lambda^2}{c}
\end{align}
$A_{ul,j}$ ist the Einstein coefficient of spontaneous emission from upper to lower energy state, $N_{u,j} $ the density of the upper state and $f_j(\lambda)$ is the line shape. The absorption coefficient $\kappa_j(\lambda)$ of this transition is given by:
\begin{align}
	\kappa_j(\lambda)  = \dfrac{h}{\lambda}  \left(  B_{lu;j} N_{l;j} -  B_{ul;j} N_{u;j} \right) f_j(\lambda)
	 \dfrac{\lambda^2}{c}
\end{align}
with the Einstein coefficients of absorption and stimulated emission:
\begin{align}
	B_{ul;j} &= \dfrac{1}{8 \pi} \dfrac{\lambda^3}{h} A_{ul;j} \;\;\; , \;\;\; \left[\dfrac{m^3}{J s^2}\right] \\
	B_{lu;j} &= \dfrac{g_u}{g_l} B_{ul;j}
\end{align}
The densities of the upper and lower states are given by the Boltzmann distribution at local temperature $T(z)$:
\begin{align}
	N_{u;j} &= N \dfrac{g_{u;j}}{Q(T)} \exp\left(- \dfrac{E_{u;j}}{k_B T} \right) \\
	N_{l;j} &= N \dfrac{g_{l;j}}{Q(T)} \exp\left(- \dfrac{E_{l;j}}{k_B T} \right)
\end{align}
$g_{u;j}$ and $g_{l;j}$ are the degeneracies of the upper and lower level respectively and Q(T) is the partition function of the $\mathrm{CO}_2$ isotope of line $j$.



\subsection{Line Shapes}

The main line broadening mechanisms in gases are natural line broadening, Doppler broadening and pressure broadening. Natural line broadening can be neglected. Pressure broadening is dominant in the denser parts of the atmosphere whereas Doppler broadening only becomes the dominant broadening mechanism in higher diluted regions of the atmosphere.

\subsubsection{Doppler Broadening}

Doppler broadened line shapes are given by a Gaussian function:
\begin{align}
	f_G(\lambda) &= \sqrt{\dfrac{\ln 2}{\pi \Delta \lambda^2}}
		\exp \left(- \dfrac{\ln 2}{\Delta \lambda^2}  \left(\lambda - \lambda_0\right)^2 \right) \\
			\int_{-\infty}^{\infty}  f_G(\lambda) d\lambda &= 1
\end{align}
with the half width at half maximum (HWHM) line width:
\begin{align}
\dfrac{\Delta \lambda}{\lambda} = \dfrac{v}{c} = \dfrac{1}{c} \sqrt{\dfrac{2 k_B T}{M}}
\end{align}
Doppler broadening is  determined by the temperature and the mass $M$ of the particles.

\subsubsection{Pressure Broadening}

Pressure broadening is caused by the collisions between molecules, in the present model mainly between $\mathrm{N}_2$ and $\mathrm{O}_2$ with $\mathrm{CO}_2$. 
The main determining factors are the concentration of the collision partners and the collision frequency. The line shapes are given by a Lorentz function:
\begin{align}
	f_L(\lambda) &= \dfrac{1}{\pi} \dfrac{\Delta \lambda}{ (\lambda - \lambda_0)^2 + \Delta \lambda^2} \\
	\int_{-\infty}^{\infty}  f_L(\lambda) d\lambda &= 1
\end{align}

Contrary to the Gaussian line shapes of Doppler broadening Lorentz functions have a much wider extend. In order to keep computation times low the Lorentz functions have to be cut at a point. To estimate the introduced error the normalized Lorentz function is integrated from $-x_p$ to $x_p$:
\begin{align}
	F(x_p) = \dfrac{1}{\pi} \int_{-x_p}^{x_p} \dfrac{1}{1 + x^2} dx = \dfrac{1}{\pi} \left(\arctan(x_p) - \arctan(-x_p)\right)
\end{align}
$F(x_p) = 0.9$ at $x_p \approx 6.3$, $0.97$ at $x_p = 20$ and $0.99$ at $x_p = 40$. In the absorption computations the limit is set at  $20\;[] \Delta \lambda$ so that approximately 3\% of the radiation power is missing. To compensate for this a background of 3\% of a moving average is besing added.

\subsubsection{Voigt Profile}

Pressure and Doppler broadening occur simulatneosly, the line shape is a convolution of Lorentz and Gauss functions which is called Voigt profile. For the sake of performance the Voight profile is approximate by a linear interpolation of Lorentz and Gauss shapes:
\begin{align}
	f(\lambda, \lambda_0) =
	\begin{cases}
		f_L(\lambda, \lambda_0) & v > 1\\
		v f_L(\lambda, \lambda_0) + (1-v) f_G(\lambda, \lambda_0) & \text{otherwise}
	\end{cases}
\end{align}
with:
\begin{align*}
	a &= \dfrac{\Delta \lambda_L}{\Delta \lambda_G \lambda_0} \\
	v &= \operatorname{max}(0.0, 1.36606 \; a - 0.47719 \; a^2 + 0.11116 \;  a^3)
\end{align*}



